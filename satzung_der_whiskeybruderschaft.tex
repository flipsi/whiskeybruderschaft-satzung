SATZUNG DER WHISKEYBRUDERSCHAFT

Fassung vom 06.04.2015



    PRÄAMBEL

Im Bewusstsein seiner Verantwortung gegenüber der Trinkkultur des menschlichen Geschlechts haben sich die Mitglieder der Whiskeybruderschaft in der Absicht dem Lebenswasser seine verdiente Ehre zu erweisen und sich für die Kunst dessen Erzeugung sowie seines Genusses zu sensibilisieren die folgenden Grundsätze gegeben.



    ARTIKEL 1
    Würde des Whiskeys und der Whiskeybrüder

(1) Die Würde des Whiskeys und die der Mitglieder der Whiskeybruderschaft sind unantastbar. Sie zu achten und zu wahren ist oberste gemeinschaftliche Aufgabe der Bruderschaft.

(2) Die Mitglieder der Whiskeybruderschaft nennen sich Whiskeybrüder. Whiskeybrüder behandeln sich gegenseitig und jede Art von Whiskey stets mit Respekt. Dies gilt für Whiskeybrüder und Whiskeys jeglicher Herkunft, Farbe, religiöser oder ideologischer Gesinnung.



    ARTIKEL 2
    Gegenstand der Bruderschaft

(1) Jeder Whiskeybruder trägt dafür Sorge, dass die Bruderschaft einen hinreichend intensiven Kontakt pflegt und gesellig ist. Das Ausmaß der Aktivitäten zu diesem Zweck liegt im Ermessen der Whiskeybrüder selbst.

(2) Gemeinschaftliche Aktivitäten sollen die Brüderlichkeit der Whiskeybrüder stärken und die Whiskeybruderschaft aufrechterhalten. Dabei darf kein Whiskeybruder ausgeschlossen werden.

(3) Zu den gemeinschaftlichen Aktivitäten gehören insbesondere die Verkostung von Whiskey, die Diskussion über Whiskey und die Aneignung von Wissen über Whiskey. Jeder Whiskeybruder konzediert die dafür notwendige Zeit, Aufmerksamkeit und Hingabe.



    ARTIKEL 3
    Bruderschaftswhiskeys

(1) Wichtiger Bestandteil der Whiskeybruderschaft sind die sogenannten Bruderschaftswhiskeys. Diese sind nicht Eigentum einzelner Mitglieder, sondern Eigentum der Bruderschaft.

(2) Whiskeybrüder sind der satzungsgemäßen Verwahrung aller Bruderschaftswhiskeys verpflichtet.

(3) Es muss stets mindestens einen Bruderschaftswhiskey geben.

(4) Ein Whiskey kann auf verschiedene Weisen zum Bruderschaftswhiskey erhoben werden: Durch Beitritt eines neuen Mitglieds (siehe Artikel 4), durch Spende eines der Mitglieder, durch bruderschaftlichen Kauf, durch Schenkung von außerhalb. Diese Liste ist nicht abschließend. Die Erhebung eines Whiskeys zum Bruderschaftswhiskey ist irreversibel.

(5) Ein Bruderschaftswhiskey darf nur in Gemeinschaft mit der Bruderschaft oder mit Einwilligung bzw. Genehmigung seiner Mitglieder konsumiert werden. Bei äußeren Umständen, die das Zusammenkommen erschweren, gestehen Whiskeybrüder sich den verantwortungsvollen Genuss auch in gegenseitiger Abwesenheit ein. Dies darf jedoch nicht überhandnehmen. Im Streitfall kommt es zu einer Schlichtung, bei der mindestens beide Streitparteien und der Präsident anwesend sind. Bei wiederholtem oder schwerwiegendem Verstoß kann ein Mitglied aus der Whiskeybruderschaft ausgeschlossen werden.



    ARTIKEL 4
    Mitgliedschaft

(1) Ein Whiskeybruder erkennt die  Satzung in ihrer Vollständigkeit an und verpflichtet sich, sein Handeln stets nach ihr auszurichten.

(2) Die Erlangung der Mitgliedschaft erfordert die ausdrückliche Anerkennung der Satzung aus freiem Willen durch den Bewerber und die einstimmige Zustimmung aller bisherigen Whiskeybrüder. Außerdem muss der Bewerber der Bruderschaft einen Bruderschaftswhiskey (siehe Artikel 3) vermachen.

(3) Ein Whiskeybruder kann seine Mitgliedschaft in der Bruderschaft freiwillig beenden. Dem hat ein Gespräch mit dem Präsidenten vorauszugehen. Ein austretender Whiskeybruder hat keinen Anspruch auf Eigentum der Whiskeybruderschaft, insbesondere nicht auf Bruderschaftswhiskeys.

(4) Wer die Grundsätze dieser Satzung wiederholt oder grob verletzt, kann von der Whiskeybruderschaft ausgeschlossen werden. Dem Ausschluss müssen mehr als die Hälfte aller Whiskeybrüder und der Präsident zustimmen.

(5) Die Mitgliedschaft wird vom Präsidenten persönlich verliehen. Einen Austritt oder Ausschluss vollzieht ebenfalls der Präsident persönlich. Der Tod beendet die Mitgliedschaft in der Whiskeybruderschaft.



    ARTIKEL 5
    Präsident der Whiskeybruderschaft

(1) Oberstes Amt der Whiskeybruderschaft ist das Amt ihres Präsidenten. Er trägt in besonderer Weise dafür Sorge, dass die Bruderschaft gepflegt und die Satzung eingehalten wird.

(2) Das Amt des Präsidenten kann nur ein Mitglied der Whiskeybruderschaft wahrnehmen.

(3) Der Präsident wird demokratisch und frei von allen Whiskeybrüdern mit einer Mehrheit von zwei Dritteln gewählt. Wenn nach zwei Wahlgängen niemand eine solche Mehrheit auf sich vereinigt, ist gewählt, wer im dritten Wahlgang die meisten Stimmen auf sich vereinigt.

(4) Die Amtszeit beträgt 1 Jahr. Jeder Whiskeybruder hat das Recht, zu einem früheren Zeitpunkt Neuwahlen einzuberufen, wenn er die satzungsgemäße Ausführung durch den Amtsträger begründet anzweifelt.

(5) Der Präsident kann Whiskeybrüder in zweckgebundene Ämter berufen und ihnen diese entziehen. Insbesondere kann dies die Verwahrung von Whiskeys, die Organisation von gemeinschaftlichen Aktivitäten, die Verwaltung der Mitglieder und Bruderschaftsdokumente sowie die Abhaltung der Wahlen betreffen.

(6) Der Präsident hat stets ein offenes Ohr für die Anliegen seiner Whiskeybrüder. Im Gegenzug räumen die Whiskeybrüder dem Präsidenten seine besondere Stellung ein und bringen ihm angemessenen Respekt entgegen.



    ARTIKEL 6
    Inkrafttreten und Gültigkeit der Satzung

(1) Die Satzung tritt am Tag ihrer Verkündigung am TT.MM.JJJJ in Kraft und gilt auf unbestimmte Zeit. Seit diesem Tag besteht die Whiskeybruderschaft.

(2) Die Satzung der Whiskeybruderschaft und damit ihr Bestand verliert ihre Gültigkeit ausschließlich durch einstimmige Auflösung oder durch Tod aller Whiskeybrüder.



    ARTIKEL 7
    Änderung der Satzung

(1) Änderungen an dieser Satzung dürfen die Grundzüge ihres Inhaltes nicht verändern.

(2) Eine Änderung der Satzung erfordert die einstimmige Einwilligung aller Whiskeybrüder. Sie muss durch den Präsidenten vollzogen und verkündet werden.



