\documentclass[a4paper,12pt]{article}

\usepackage[utf8]{inputenc}
\usepackage[ngerman]{babel}

\usepackage{lmodern} % Latin Modern typeface (font)
\usepackage[T1]{fontenc} % T1 font encoding for high quality font output


\usepackage{enumitem}
\setenumerate[0]{label= (\arabic*)}


\title{Satzung der Whiskeybruderschaft}
% \author{Philipp Moers}


% SWAP COMMENTING WHEN COMPILING FINAL/PUBLISHED VERSIONS!
\date{Fassung vom 15.05.2015}
% \date{Entwurf vom \today}




\begin{document}

\maketitle

\clearpage

\begin{center}
\section*{Präambel}
Im Bewusstsein seiner Verantwortung gegenüber der Trinkkultur des menschlichen Geschlechts haben
sich die Mitglieder der Whiskeybruderschaft in der Absicht dem Lebenswasser seine verdiente Ehre zu
erweisen und sich für die Kunst dessen Erzeugung sowie seines Genusses zu sensibilisieren die
folgenden Grundsätze gegeben.
\end{center}

\clearpage



\section*{Artikel 1: Würde des Whiskeys und der Whiskeybrüder}

\begin{enumerate}

\item Die Würde des Whiskeys und die der Mitglieder der Whiskeybruderschaft sind unantastbar. Sie zu
  achten und zu wahren ist oberste gemeinschaftliche Aufgabe der Bruderschaft.

\item Die Mitglieder der Whiskeybruderschaft nennen sich Whiskeybrüder. Whiskeybrüder behandeln sich
  gegenseitig und jede Art von Whiskey stets mit Respekt. Dies gilt für Whiskeybrüder und Whiskeys
  jeglicher Herkunft, Farbe, religiöser oder ideologischer Gesinnung.

\end{enumerate}



\section*{Artikel 2: Gegenstand der Bruderschaft}

\begin{enumerate}

\item Jeder Whiskeybruder trägt dafür Sorge, dass die Bruderschaft einen hinreichend intensiven
  Kontakt pflegt und gesellig ist. Das Ausmaß der Aktivitäten zu diesem Zweck liegt im Ermessen der
  Whiskeybrüder selbst.

\item Gemeinschaftliche Aktivitäten sollen die Brüderlichkeit der Whiskeybrüder stärken und die
  Whiskeybruderschaft aufrechterhalten. Dabei darf kein Whiskeybruder ausgeschlossen werden.

\item Zu den gemeinschaftlichen Aktivitäten gehören insbesondere die Verkostung von Whiskey, die
  Diskussion über Whiskey und die Aneignung von Wissen über Whiskey. Jeder Whiskeybruder konzediert
  die dafür notwendige Zeit, Aufmerksamkeit und Hingabe.

\end{enumerate}



\section*{Artikel 3: Bruderschaftswhiskeys}

\begin{enumerate}

\item Wichtiger Bestandteil der Whiskeybruderschaft sind die sogenannten Bruderschaftswhiskeys.
  Diese sind nicht Eigentum einzelner Mitglieder, sondern Eigentum der Bruderschaft.

\item Ein Bruderschaftswhiskey muss gehobenen Ansprüchen genügen. Das schließt günstige, im
  durchschnittlichen Supermarkt erhältliche Whiskeys als Bruderschaftswhiskey aus.

\item Whiskeybrüder sind der satzungsgemäßen Verwahrung aller Bruderschaftswhiskeys verpflichtet.

\item Es muss stets mindestens einen Bruderschaftswhiskey geben.

\item Ein Whiskey kann auf verschiedene Weisen zum Bruderschaftswhiskey erhoben werden: Durch
  Beitritt eines neuen Mitglieds (siehe Artikel 4), durch Spende eines der Mitglieder, durch
  bruderschaftlichen Kauf, durch Schenkung von außerhalb. Diese Liste ist nicht abschließend. Die
  Erhebung eines Whiskeys zum Bruderschaftswhiskey ist irreversibel.

\item Ein Bruderschaftswhiskey darf nur in Gemeinschaft mit der Bruderschaft oder mit Einwilligung
  bzw. Genehmigung seiner Mitglieder geöffnet und konsumiert werden. Bei äußeren Umständen, die das
  Zusammenkommen erschweren, gestehen Whiskeybrüder sich den verantwortungsvollen Genuss auch in
  gegenseitiger Abwesenheit ein. Dies darf jedoch nicht überhandnehmen. Im Streitfall kommt es zu
  einer Schlichtung, bei der mindestens beide Streitparteien und der Präsident anwesend sind. Bei
  wiederholtem oder schwerwiegendem Verstoß kann ein Mitglied aus der Whiskeybruderschaft
  ausgeschlossen werden.

\end{enumerate}



\section*{Artikel 4: Mitgliedschaft}

\begin{enumerate}

\item Ein Whiskeybruder erkennt die  Satzung in ihrer Vollständigkeit an und verpflichtet sich, sein
  Handeln stets nach ihr auszurichten.

\item Die Erlangung der Mitgliedschaft erfordert die ausdrückliche Anerkennung der Satzung aus
  freiem Willen durch den Bewerber und die einstimmige Zustimmung aller bisherigen Whiskeybrüder.
  Außerdem muss der Bewerber der Bruderschaft einen Bruderschaftswhiskey (siehe Artikel 3)
  vermachen. Dieser Whiskey muss originalversiegelt sein. Whiskeybrüder können ihre Stimme von dem
  Whiskey des Bewerbers abhängig machen.

\item Ein Whiskeybruder kann seine Mitgliedschaft in der Bruderschaft freiwillig beenden. Dem hat
  ein Gespräch mit dem Präsidenten vorauszugehen. Ein austretender Whiskeybruder hat keinen Anspruch
  auf Eigentum der Whiskeybruderschaft, insbesondere nicht auf Bruderschaftswhiskeys.

\item Wer die Grundsätze dieser Satzung wiederholt oder grob verletzt, kann von der
  Whiskeybruderschaft ausgeschlossen werden. Dem Ausschluss müssen mehr als die Hälfte aller
  Whiskeybrüder und der Präsident zustimmen.

\item Die Mitgliedschaft wird vom Präsidenten persönlich verliehen. Bei der Aufnahmezeremonie wird
  die Satzung verlesen. Einen Austritt oder Ausschluss vollzieht ebenfalls der Präsident persönlich.

\item Ein Whiskeybruder kann auf eigenen Wunsch durch den Präsidenten in den Zustand der Inaktivität
  versetzt werden. Ein inaktiver Whiskeybruder ist von jeglichen Rechten und Pflichten gemäß der
  Satzung befreit. Insbesondere darf er keinen Bruderschaftswhiskey konsumieren. Die Inaktivität
  kann nur auf Wunsch des Inaktiven und durch den Präsidenten beendet werden.

\item Der Tod eines Whiskeybruders versetzt ihn in den Zustand der Inaktivität. Verstorbenen
  Whiskeybrüdern wird von Zeit zu Zeit beim Genuss von Bruderschaftswhiskey gedacht.

\end{enumerate}



\section*{Artikel 5: Präsident der Whiskeybruderschaft}

\begin{enumerate}

\item Oberstes Amt der Whiskeybruderschaft ist das Amt ihres Präsidenten. Er trägt in besonderer
  Weise dafür Sorge, dass die Bruderschaft gepflegt und die Satzung eingehalten wird.

\item Das Amt des Präsidenten kann nur ein Mitglied der Whiskeybruderschaft wahrnehmen.

\item Der Präsident wird demokratisch und frei von allen Whiskeybrüdern mit einer Mehrheit von zwei
  Dritteln gewählt. Wenn nach zwei Wahlgängen niemand eine solche Mehrheit auf sich vereinigt, ist
  gewählt, wer im dritten Wahlgang die meisten Stimmen auf sich vereinigt.

\item Die Amtszeit beträgt 1 Jahr. Jeder Whiskeybruder hat das Recht, zu einem früheren Zeitpunkt
  Neuwahlen einzuberufen, wenn er die satzungsgemäße Ausführung durch den Amtsträger begründet
  anzweifelt.

\item Der Präsident kann Whiskeybrüder in zweckgebundene Ämter berufen und ihnen diese entziehen.
  Insbesondere kann dies die Verwahrung von Whiskeys, die Organisation von gemeinschaftlichen
  Aktivitäten, die Verwaltung der Mitglieder und Bruderschaftsdokumente sowie die Abhaltung der
  Wahlen betreffen.

\item Der Präsident hat stets ein offenes Ohr für die Anliegen seiner Whiskeybrüder. Im Gegenzug
  Respekt entgegen.

\end{enumerate}



\section*{Artikel 6: Inkrafttreten und Gültigkeit der Satzung}

\begin{enumerate}

\item Die Satzung tritt am Tag ihrer Verkündigung am 15.05.2015 in Kraft und gilt auf unbestimmte
  Zeit. Seit diesem Tag besteht die Whiskeybruderschaft.

\item Die Satzung der Whiskeybruderschaft und damit ihr Bestand verliert ihre Gültigkeit
  ausschließlich durch einstimmige Auflösung oder durch Tod aller Whiskeybrüder.

\end{enumerate}



\section*{Artikel 7: Änderung der Satzung}

\begin{enumerate}

\item Änderungen an dieser Satzung dürfen die Grundzüge ihres Inhaltes nicht verändern.

\item Eine Änderung der Satzung erfordert die einstimmige Einwilligung aller Whiskeybrüder. Sie muss
  durch den Präsidenten vollzogen und verkündet werden.

\end{enumerate}



\end{document}



% modeline for vim
\begin{comment} vim:tw=100 \end{comment}


